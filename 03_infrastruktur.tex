\section{Aufbau der Infrastruktur}

Das Spiel sowie auch der Levelgenerator werden mittels der Programmiersprache Java 8 in der Entwicklungsumgebung IntelliJ IDEA entwickelt. Beide Programme werden jeweils in einem eigenen IntelliJ Projekt implementiert.
Dadurch ist eine strikte Unabhängigkeit zwischen dem Levelgenerator und dem Spiel gewährleistet.

Um die Teamarbeit effizient zu gestalten, wird das Versionierungswerkzeug Git\footnote{https://git-scm.com} eingesetzt.
Es ermöglicht die Zusammenarbeit über die Plattform GitHub\footnote{https://github.com}, auf der verschiedene nützliche Features angeboten werden.
Sehr häufige Verwendung in diesem Projekt findet die Erzeugung und Zusammenführung \textit{von Pull Requests}.
Nachdem ein Entwickler ein Feature entwickelt oder einen Fehler behoben hat, wird so ein Pull Request erstellt.
Ein weiteres Teammitglied überprüft diese Codeänderung und merged sie, bei positivem Ergebnis, auf die Master-Branch.

Eine weitere wichtige GitHub Funktion besteht im Erstellen von sogenannten \textit{Issues}.
Dies sind repository-interne Forumseinträge, die bestimmte projektbezogene Themen behandeln sollen.
Es können durch Labels Rubriken festgelegt werden, wie zum Beispiel ``Bug`` oder ``Question``.
So erhält das Projekt eine Struktur und Entscheidungen bzw. Hinweise, die für das gesamte Team wichtig sind, werden an zentraler Stelle veröffentlicht und zur Diskussion freigegeben.

Für das Bauen und die Verwaltung von Abhängigkeiten zu Bibliotheken und Frameworks wird im Spieleprojekt das Tool Maven verwendet.
Im Generatorprojekt dagegen besteht keine Notwendigkeit für ein solches Tool, weshalb hier Maven nicht eingesetzt wird.

Alle im Spiel verwendeten Grafiken wurden selbst mit Hilfe der Software Photoshop im Pixelart Stil erstellt. Bei dieser Art von Bilderstellung wird Pixel für Pixel eingefärbt und mit anderen Pixeln zusammengesetzt. Das Ergebnis ergibt dann ein Bild, das verwendet werden kann.