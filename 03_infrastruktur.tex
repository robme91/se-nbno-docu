\section{Aufbau der Infrastruktur}

Das Spiel sowie auch der Levelgenerator werden mittels der Programmiersprache Java 8 entwickelt. 
Als Entwicklungsumgebung dient das IntelliJ IDEA System. Beide Programme werden jeweils in einem eigenen IntelliJ Projekt implementiert. Dadurch ist eine strikte Unabhängigkeit zwischen dem Levelgenerator und dem Spiel gewährleistet.\newline
Um die Teamarbeit effizient zu gestalten, wird das Versionierungswerkzeug GIT eingesetzt. Es ermöglicht die Zusammenarbeit über die Plattform GitHub, auf der verschiedene nützliche Features angeboten werden. Sehr häufige Verwendung in diesem Projekt findet die Erzeugung und Zusammenführung von Pull Requests. Nachdem ein Entwickler ein Feature entwickelt oder einen Fehler behoben hat, wird so ein Pull Request erstellt. Ein weiteres Teammitglied überprüft diese Codeänderung und merged sie, bei positivem Ergebnis, auf die Master Branch.\newline
Eine weitere wichtige GitHub Funktion besteht im Erstellen von sogenannte Issues. Dies sind Repository interne Forumseinträge, die bestimmte Projektbezogene Themen behandeln sollen. Es können durch Labels Rubriken festgelegt werden, wie zum Beispiel ``Bug'' oder ``Question''. So erhält das Projekt eine Struktur und Entscheidungen bzw. Hinweise die für das gesamte Team wichtig sind, werden an zentraler Stelle veröffentlicht und zur Diskussion freigegeben.\newline
Für das Bauen und die Abhängigkeiten Automatisierung von Bibliotheken und Frameworks wird im Spiele Projekt das Tool Maven verwendet. Im Generator Projekt dagegen besteht keine Notwendigkeit für ein solches Tool, weshalb hier Maven nicht eingesetzt wird.\newline
