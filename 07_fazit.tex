\section{Fazit}

Das Ziel dieses Software Engineering Projekts war es, durch einen ansprechenden Rahmen Codegenerierung anzuwenden und besser kennenzulernen.
In diesem Projekt wurde die Codegenerierung bei der Levelgenerierung für ein selbstgeschriebenes 2D Spiel namens NBNO angewendet.

Das Spiel hält einen Spieler dazu an, stetig sein Bier aufzufüllen und dabei schadenden Gegnern auszuweichen.
Das Spielprinzip und die Umsetzung sind sehr gelungen und ein Spaßfaktor ist dem Spiel nicht abzusprechen.

Die Level dieses Spiels werden im Levelgenerator erzeugt, der aus einer textbasierten nbno-Datei eine Javaklasse generiert, die in das Spiel eingebunden werden kann.

Beide Projektteile sind gelungen und somit wurde das Projektziel erreicht.
Die verwendeten Technologien Slick2D und Antlr waren für die Umsetzung sehr nützlich und hilfreich.
Die Arbeitsweise mit diesen Werkzeugen ist als gelungen anzusehen, weshalb bei ähnlichen Problemstellungen einer Wiederverwendung dieser Tools wahrscheinlich ist.