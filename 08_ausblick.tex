\section{Ausblick}

Im folgenden Abschnitt finden sich Ideen und Anregungen zur Weiterentwicklung des Spiels bzw. der NBNO-DSL.

\subsubsection{Dynamische Spielfeldgröße}
Wie in Listing \ref{lst:codegen-mapCheck} ersichtlich, ist die aktuelle Spielfeldgröße auf 20x25 Blöcke beschränkt.
Wünschenswert wäre eine freie Spielfeldgröße.
Das Fenster der Anwendung sollte dabei eine anpassbare Größe erhalten und ggf. scrollen.

\subsubsection{Mehr Blöcke}
Das hierarchisch aufgebaute Modell ermöglicht die Einbindung neuer Blocktypen.
Um das Spiel abwechslungsreicher zu gestalten wären weitere Blocktypen, wie beispielsweise Parkbänke welche die Zeit schneller vergehen lassen, interessant.

\subsubsection{Mehr Freiheit bei der Levelerzeugung}
Derzeit sind viele, für die Spieldynamik relevante, Parameter noch statisch.
Dazu gehören unter anderem die Zeit, die ein Späti braucht um wieder verfügbar zu sein.
Um Levelentwicklern mehr Freiheit zu geben, ist eine (optionale) Auslagerung aller spielrelevanten Parameter in die NBNO-DSL sinnvoll.